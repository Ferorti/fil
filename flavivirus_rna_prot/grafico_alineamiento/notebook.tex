
% Default to the notebook output style

    


% Inherit from the specified cell style.




    
\documentclass[11pt]{article}

    
    
    \usepackage[T1]{fontenc}
    % Nicer default font (+ math font) than Computer Modern for most use cases
    \usepackage{mathpazo}

    % Basic figure setup, for now with no caption control since it's done
    % automatically by Pandoc (which extracts ![](path) syntax from Markdown).
    \usepackage{graphicx}
    % We will generate all images so they have a width \maxwidth. This means
    % that they will get their normal width if they fit onto the page, but
    % are scaled down if they would overflow the margins.
    \makeatletter
    \def\maxwidth{\ifdim\Gin@nat@width>\linewidth\linewidth
    \else\Gin@nat@width\fi}
    \makeatother
    \let\Oldincludegraphics\includegraphics
    % Set max figure width to be 80% of text width, for now hardcoded.
    \renewcommand{\includegraphics}[1]{\Oldincludegraphics[width=.8\maxwidth]{#1}}
    % Ensure that by default, figures have no caption (until we provide a
    % proper Figure object with a Caption API and a way to capture that
    % in the conversion process - todo).
    \usepackage{caption}
    \DeclareCaptionLabelFormat{nolabel}{}
    \captionsetup{labelformat=nolabel}

    \usepackage{adjustbox} % Used to constrain images to a maximum size 
    \usepackage{xcolor} % Allow colors to be defined
    \usepackage{enumerate} % Needed for markdown enumerations to work
    \usepackage{geometry} % Used to adjust the document margins
    \usepackage{amsmath} % Equations
    \usepackage{amssymb} % Equations
    \usepackage{textcomp} % defines textquotesingle
    % Hack from http://tex.stackexchange.com/a/47451/13684:
    \AtBeginDocument{%
        \def\PYZsq{\textquotesingle}% Upright quotes in Pygmentized code
    }
    \usepackage{upquote} % Upright quotes for verbatim code
    \usepackage{eurosym} % defines \euro
    \usepackage[mathletters]{ucs} % Extended unicode (utf-8) support
    \usepackage[utf8x]{inputenc} % Allow utf-8 characters in the tex document
    \usepackage{fancyvrb} % verbatim replacement that allows latex
    \usepackage{grffile} % extends the file name processing of package graphics 
                         % to support a larger range 
    % The hyperref package gives us a pdf with properly built
    % internal navigation ('pdf bookmarks' for the table of contents,
    % internal cross-reference links, web links for URLs, etc.)
    \usepackage{hyperref}
    \usepackage{longtable} % longtable support required by pandoc >1.10
    \usepackage{booktabs}  % table support for pandoc > 1.12.2
    \usepackage[inline]{enumitem} % IRkernel/repr support (it uses the enumerate* environment)
    \usepackage[normalem]{ulem} % ulem is needed to support strikethroughs (\sout)
                                % normalem makes italics be italics, not underlines
    

    
    
    % Colors for the hyperref package
    \definecolor{urlcolor}{rgb}{0,.145,.698}
    \definecolor{linkcolor}{rgb}{.71,0.21,0.01}
    \definecolor{citecolor}{rgb}{.12,.54,.11}

    % ANSI colors
    \definecolor{ansi-black}{HTML}{3E424D}
    \definecolor{ansi-black-intense}{HTML}{282C36}
    \definecolor{ansi-red}{HTML}{E75C58}
    \definecolor{ansi-red-intense}{HTML}{B22B31}
    \definecolor{ansi-green}{HTML}{00A250}
    \definecolor{ansi-green-intense}{HTML}{007427}
    \definecolor{ansi-yellow}{HTML}{DDB62B}
    \definecolor{ansi-yellow-intense}{HTML}{B27D12}
    \definecolor{ansi-blue}{HTML}{208FFB}
    \definecolor{ansi-blue-intense}{HTML}{0065CA}
    \definecolor{ansi-magenta}{HTML}{D160C4}
    \definecolor{ansi-magenta-intense}{HTML}{A03196}
    \definecolor{ansi-cyan}{HTML}{60C6C8}
    \definecolor{ansi-cyan-intense}{HTML}{258F8F}
    \definecolor{ansi-white}{HTML}{C5C1B4}
    \definecolor{ansi-white-intense}{HTML}{A1A6B2}

    % commands and environments needed by pandoc snippets
    % extracted from the output of `pandoc -s`
    \providecommand{\tightlist}{%
      \setlength{\itemsep}{0pt}\setlength{\parskip}{0pt}}
    \DefineVerbatimEnvironment{Highlighting}{Verbatim}{commandchars=\\\{\}}
    % Add ',fontsize=\small' for more characters per line
    \newenvironment{Shaded}{}{}
    \newcommand{\KeywordTok}[1]{\textcolor[rgb]{0.00,0.44,0.13}{\textbf{{#1}}}}
    \newcommand{\DataTypeTok}[1]{\textcolor[rgb]{0.56,0.13,0.00}{{#1}}}
    \newcommand{\DecValTok}[1]{\textcolor[rgb]{0.25,0.63,0.44}{{#1}}}
    \newcommand{\BaseNTok}[1]{\textcolor[rgb]{0.25,0.63,0.44}{{#1}}}
    \newcommand{\FloatTok}[1]{\textcolor[rgb]{0.25,0.63,0.44}{{#1}}}
    \newcommand{\CharTok}[1]{\textcolor[rgb]{0.25,0.44,0.63}{{#1}}}
    \newcommand{\StringTok}[1]{\textcolor[rgb]{0.25,0.44,0.63}{{#1}}}
    \newcommand{\CommentTok}[1]{\textcolor[rgb]{0.38,0.63,0.69}{\textit{{#1}}}}
    \newcommand{\OtherTok}[1]{\textcolor[rgb]{0.00,0.44,0.13}{{#1}}}
    \newcommand{\AlertTok}[1]{\textcolor[rgb]{1.00,0.00,0.00}{\textbf{{#1}}}}
    \newcommand{\FunctionTok}[1]{\textcolor[rgb]{0.02,0.16,0.49}{{#1}}}
    \newcommand{\RegionMarkerTok}[1]{{#1}}
    \newcommand{\ErrorTok}[1]{\textcolor[rgb]{1.00,0.00,0.00}{\textbf{{#1}}}}
    \newcommand{\NormalTok}[1]{{#1}}
    
    % Additional commands for more recent versions of Pandoc
    \newcommand{\ConstantTok}[1]{\textcolor[rgb]{0.53,0.00,0.00}{{#1}}}
    \newcommand{\SpecialCharTok}[1]{\textcolor[rgb]{0.25,0.44,0.63}{{#1}}}
    \newcommand{\VerbatimStringTok}[1]{\textcolor[rgb]{0.25,0.44,0.63}{{#1}}}
    \newcommand{\SpecialStringTok}[1]{\textcolor[rgb]{0.73,0.40,0.53}{{#1}}}
    \newcommand{\ImportTok}[1]{{#1}}
    \newcommand{\DocumentationTok}[1]{\textcolor[rgb]{0.73,0.13,0.13}{\textit{{#1}}}}
    \newcommand{\AnnotationTok}[1]{\textcolor[rgb]{0.38,0.63,0.69}{\textbf{\textit{{#1}}}}}
    \newcommand{\CommentVarTok}[1]{\textcolor[rgb]{0.38,0.63,0.69}{\textbf{\textit{{#1}}}}}
    \newcommand{\VariableTok}[1]{\textcolor[rgb]{0.10,0.09,0.49}{{#1}}}
    \newcommand{\ControlFlowTok}[1]{\textcolor[rgb]{0.00,0.44,0.13}{\textbf{{#1}}}}
    \newcommand{\OperatorTok}[1]{\textcolor[rgb]{0.40,0.40,0.40}{{#1}}}
    \newcommand{\BuiltInTok}[1]{{#1}}
    \newcommand{\ExtensionTok}[1]{{#1}}
    \newcommand{\PreprocessorTok}[1]{\textcolor[rgb]{0.74,0.48,0.00}{{#1}}}
    \newcommand{\AttributeTok}[1]{\textcolor[rgb]{0.49,0.56,0.16}{{#1}}}
    \newcommand{\InformationTok}[1]{\textcolor[rgb]{0.38,0.63,0.69}{\textbf{\textit{{#1}}}}}
    \newcommand{\WarningTok}[1]{\textcolor[rgb]{0.38,0.63,0.69}{\textbf{\textit{{#1}}}}}
    
    
    % Define a nice break command that doesn't care if a line doesn't already
    % exist.
    \def\br{\hspace*{\fill} \\* }
    % Math Jax compatability definitions
    \def\gt{>}
    \def\lt{<}
    % Document parameters
    \title{Visualizador}
    
    
    

    % Pygments definitions
    
\makeatletter
\def\PY@reset{\let\PY@it=\relax \let\PY@bf=\relax%
    \let\PY@ul=\relax \let\PY@tc=\relax%
    \let\PY@bc=\relax \let\PY@ff=\relax}
\def\PY@tok#1{\csname PY@tok@#1\endcsname}
\def\PY@toks#1+{\ifx\relax#1\empty\else%
    \PY@tok{#1}\expandafter\PY@toks\fi}
\def\PY@do#1{\PY@bc{\PY@tc{\PY@ul{%
    \PY@it{\PY@bf{\PY@ff{#1}}}}}}}
\def\PY#1#2{\PY@reset\PY@toks#1+\relax+\PY@do{#2}}

\expandafter\def\csname PY@tok@gd\endcsname{\def\PY@tc##1{\textcolor[rgb]{0.63,0.00,0.00}{##1}}}
\expandafter\def\csname PY@tok@gu\endcsname{\let\PY@bf=\textbf\def\PY@tc##1{\textcolor[rgb]{0.50,0.00,0.50}{##1}}}
\expandafter\def\csname PY@tok@gt\endcsname{\def\PY@tc##1{\textcolor[rgb]{0.00,0.27,0.87}{##1}}}
\expandafter\def\csname PY@tok@gs\endcsname{\let\PY@bf=\textbf}
\expandafter\def\csname PY@tok@gr\endcsname{\def\PY@tc##1{\textcolor[rgb]{1.00,0.00,0.00}{##1}}}
\expandafter\def\csname PY@tok@cm\endcsname{\let\PY@it=\textit\def\PY@tc##1{\textcolor[rgb]{0.25,0.50,0.50}{##1}}}
\expandafter\def\csname PY@tok@vg\endcsname{\def\PY@tc##1{\textcolor[rgb]{0.10,0.09,0.49}{##1}}}
\expandafter\def\csname PY@tok@vi\endcsname{\def\PY@tc##1{\textcolor[rgb]{0.10,0.09,0.49}{##1}}}
\expandafter\def\csname PY@tok@vm\endcsname{\def\PY@tc##1{\textcolor[rgb]{0.10,0.09,0.49}{##1}}}
\expandafter\def\csname PY@tok@mh\endcsname{\def\PY@tc##1{\textcolor[rgb]{0.40,0.40,0.40}{##1}}}
\expandafter\def\csname PY@tok@cs\endcsname{\let\PY@it=\textit\def\PY@tc##1{\textcolor[rgb]{0.25,0.50,0.50}{##1}}}
\expandafter\def\csname PY@tok@ge\endcsname{\let\PY@it=\textit}
\expandafter\def\csname PY@tok@vc\endcsname{\def\PY@tc##1{\textcolor[rgb]{0.10,0.09,0.49}{##1}}}
\expandafter\def\csname PY@tok@il\endcsname{\def\PY@tc##1{\textcolor[rgb]{0.40,0.40,0.40}{##1}}}
\expandafter\def\csname PY@tok@go\endcsname{\def\PY@tc##1{\textcolor[rgb]{0.53,0.53,0.53}{##1}}}
\expandafter\def\csname PY@tok@cp\endcsname{\def\PY@tc##1{\textcolor[rgb]{0.74,0.48,0.00}{##1}}}
\expandafter\def\csname PY@tok@gi\endcsname{\def\PY@tc##1{\textcolor[rgb]{0.00,0.63,0.00}{##1}}}
\expandafter\def\csname PY@tok@gh\endcsname{\let\PY@bf=\textbf\def\PY@tc##1{\textcolor[rgb]{0.00,0.00,0.50}{##1}}}
\expandafter\def\csname PY@tok@ni\endcsname{\let\PY@bf=\textbf\def\PY@tc##1{\textcolor[rgb]{0.60,0.60,0.60}{##1}}}
\expandafter\def\csname PY@tok@nl\endcsname{\def\PY@tc##1{\textcolor[rgb]{0.63,0.63,0.00}{##1}}}
\expandafter\def\csname PY@tok@nn\endcsname{\let\PY@bf=\textbf\def\PY@tc##1{\textcolor[rgb]{0.00,0.00,1.00}{##1}}}
\expandafter\def\csname PY@tok@no\endcsname{\def\PY@tc##1{\textcolor[rgb]{0.53,0.00,0.00}{##1}}}
\expandafter\def\csname PY@tok@na\endcsname{\def\PY@tc##1{\textcolor[rgb]{0.49,0.56,0.16}{##1}}}
\expandafter\def\csname PY@tok@nb\endcsname{\def\PY@tc##1{\textcolor[rgb]{0.00,0.50,0.00}{##1}}}
\expandafter\def\csname PY@tok@nc\endcsname{\let\PY@bf=\textbf\def\PY@tc##1{\textcolor[rgb]{0.00,0.00,1.00}{##1}}}
\expandafter\def\csname PY@tok@nd\endcsname{\def\PY@tc##1{\textcolor[rgb]{0.67,0.13,1.00}{##1}}}
\expandafter\def\csname PY@tok@ne\endcsname{\let\PY@bf=\textbf\def\PY@tc##1{\textcolor[rgb]{0.82,0.25,0.23}{##1}}}
\expandafter\def\csname PY@tok@nf\endcsname{\def\PY@tc##1{\textcolor[rgb]{0.00,0.00,1.00}{##1}}}
\expandafter\def\csname PY@tok@si\endcsname{\let\PY@bf=\textbf\def\PY@tc##1{\textcolor[rgb]{0.73,0.40,0.53}{##1}}}
\expandafter\def\csname PY@tok@s2\endcsname{\def\PY@tc##1{\textcolor[rgb]{0.73,0.13,0.13}{##1}}}
\expandafter\def\csname PY@tok@nt\endcsname{\let\PY@bf=\textbf\def\PY@tc##1{\textcolor[rgb]{0.00,0.50,0.00}{##1}}}
\expandafter\def\csname PY@tok@nv\endcsname{\def\PY@tc##1{\textcolor[rgb]{0.10,0.09,0.49}{##1}}}
\expandafter\def\csname PY@tok@s1\endcsname{\def\PY@tc##1{\textcolor[rgb]{0.73,0.13,0.13}{##1}}}
\expandafter\def\csname PY@tok@dl\endcsname{\def\PY@tc##1{\textcolor[rgb]{0.73,0.13,0.13}{##1}}}
\expandafter\def\csname PY@tok@ch\endcsname{\let\PY@it=\textit\def\PY@tc##1{\textcolor[rgb]{0.25,0.50,0.50}{##1}}}
\expandafter\def\csname PY@tok@m\endcsname{\def\PY@tc##1{\textcolor[rgb]{0.40,0.40,0.40}{##1}}}
\expandafter\def\csname PY@tok@gp\endcsname{\let\PY@bf=\textbf\def\PY@tc##1{\textcolor[rgb]{0.00,0.00,0.50}{##1}}}
\expandafter\def\csname PY@tok@sh\endcsname{\def\PY@tc##1{\textcolor[rgb]{0.73,0.13,0.13}{##1}}}
\expandafter\def\csname PY@tok@ow\endcsname{\let\PY@bf=\textbf\def\PY@tc##1{\textcolor[rgb]{0.67,0.13,1.00}{##1}}}
\expandafter\def\csname PY@tok@sx\endcsname{\def\PY@tc##1{\textcolor[rgb]{0.00,0.50,0.00}{##1}}}
\expandafter\def\csname PY@tok@bp\endcsname{\def\PY@tc##1{\textcolor[rgb]{0.00,0.50,0.00}{##1}}}
\expandafter\def\csname PY@tok@c1\endcsname{\let\PY@it=\textit\def\PY@tc##1{\textcolor[rgb]{0.25,0.50,0.50}{##1}}}
\expandafter\def\csname PY@tok@fm\endcsname{\def\PY@tc##1{\textcolor[rgb]{0.00,0.00,1.00}{##1}}}
\expandafter\def\csname PY@tok@o\endcsname{\def\PY@tc##1{\textcolor[rgb]{0.40,0.40,0.40}{##1}}}
\expandafter\def\csname PY@tok@kc\endcsname{\let\PY@bf=\textbf\def\PY@tc##1{\textcolor[rgb]{0.00,0.50,0.00}{##1}}}
\expandafter\def\csname PY@tok@c\endcsname{\let\PY@it=\textit\def\PY@tc##1{\textcolor[rgb]{0.25,0.50,0.50}{##1}}}
\expandafter\def\csname PY@tok@mf\endcsname{\def\PY@tc##1{\textcolor[rgb]{0.40,0.40,0.40}{##1}}}
\expandafter\def\csname PY@tok@err\endcsname{\def\PY@bc##1{\setlength{\fboxsep}{0pt}\fcolorbox[rgb]{1.00,0.00,0.00}{1,1,1}{\strut ##1}}}
\expandafter\def\csname PY@tok@mb\endcsname{\def\PY@tc##1{\textcolor[rgb]{0.40,0.40,0.40}{##1}}}
\expandafter\def\csname PY@tok@ss\endcsname{\def\PY@tc##1{\textcolor[rgb]{0.10,0.09,0.49}{##1}}}
\expandafter\def\csname PY@tok@sr\endcsname{\def\PY@tc##1{\textcolor[rgb]{0.73,0.40,0.53}{##1}}}
\expandafter\def\csname PY@tok@mo\endcsname{\def\PY@tc##1{\textcolor[rgb]{0.40,0.40,0.40}{##1}}}
\expandafter\def\csname PY@tok@kd\endcsname{\let\PY@bf=\textbf\def\PY@tc##1{\textcolor[rgb]{0.00,0.50,0.00}{##1}}}
\expandafter\def\csname PY@tok@mi\endcsname{\def\PY@tc##1{\textcolor[rgb]{0.40,0.40,0.40}{##1}}}
\expandafter\def\csname PY@tok@kn\endcsname{\let\PY@bf=\textbf\def\PY@tc##1{\textcolor[rgb]{0.00,0.50,0.00}{##1}}}
\expandafter\def\csname PY@tok@cpf\endcsname{\let\PY@it=\textit\def\PY@tc##1{\textcolor[rgb]{0.25,0.50,0.50}{##1}}}
\expandafter\def\csname PY@tok@kr\endcsname{\let\PY@bf=\textbf\def\PY@tc##1{\textcolor[rgb]{0.00,0.50,0.00}{##1}}}
\expandafter\def\csname PY@tok@s\endcsname{\def\PY@tc##1{\textcolor[rgb]{0.73,0.13,0.13}{##1}}}
\expandafter\def\csname PY@tok@kp\endcsname{\def\PY@tc##1{\textcolor[rgb]{0.00,0.50,0.00}{##1}}}
\expandafter\def\csname PY@tok@w\endcsname{\def\PY@tc##1{\textcolor[rgb]{0.73,0.73,0.73}{##1}}}
\expandafter\def\csname PY@tok@kt\endcsname{\def\PY@tc##1{\textcolor[rgb]{0.69,0.00,0.25}{##1}}}
\expandafter\def\csname PY@tok@sc\endcsname{\def\PY@tc##1{\textcolor[rgb]{0.73,0.13,0.13}{##1}}}
\expandafter\def\csname PY@tok@sb\endcsname{\def\PY@tc##1{\textcolor[rgb]{0.73,0.13,0.13}{##1}}}
\expandafter\def\csname PY@tok@sa\endcsname{\def\PY@tc##1{\textcolor[rgb]{0.73,0.13,0.13}{##1}}}
\expandafter\def\csname PY@tok@k\endcsname{\let\PY@bf=\textbf\def\PY@tc##1{\textcolor[rgb]{0.00,0.50,0.00}{##1}}}
\expandafter\def\csname PY@tok@se\endcsname{\let\PY@bf=\textbf\def\PY@tc##1{\textcolor[rgb]{0.73,0.40,0.13}{##1}}}
\expandafter\def\csname PY@tok@sd\endcsname{\let\PY@it=\textit\def\PY@tc##1{\textcolor[rgb]{0.73,0.13,0.13}{##1}}}

\def\PYZbs{\char`\\}
\def\PYZus{\char`\_}
\def\PYZob{\char`\{}
\def\PYZcb{\char`\}}
\def\PYZca{\char`\^}
\def\PYZam{\char`\&}
\def\PYZlt{\char`\<}
\def\PYZgt{\char`\>}
\def\PYZsh{\char`\#}
\def\PYZpc{\char`\%}
\def\PYZdl{\char`\$}
\def\PYZhy{\char`\-}
\def\PYZsq{\char`\'}
\def\PYZdq{\char`\"}
\def\PYZti{\char`\~}
% for compatibility with earlier versions
\def\PYZat{@}
\def\PYZlb{[}
\def\PYZrb{]}
\makeatother


    % Exact colors from NB
    \definecolor{incolor}{rgb}{0.0, 0.0, 0.5}
    \definecolor{outcolor}{rgb}{0.545, 0.0, 0.0}



    
    % Prevent overflowing lines due to hard-to-break entities
    \sloppy 
    % Setup hyperref package
    \hypersetup{
      breaklinks=true,  % so long urls are correctly broken across lines
      colorlinks=true,
      urlcolor=urlcolor,
      linkcolor=linkcolor,
      citecolor=citecolor,
      }
    % Slightly bigger margins than the latex defaults
    
    \geometry{verbose,tmargin=1in,bmargin=1in,lmargin=1in,rmargin=1in}
    
    

    \begin{document}
    
    
    \maketitle
    
    

    
    \section{Visualizador de zonas complementarias 5' RNA -
Prot}\label{visualizador-de-zonas-complementarias-5-rna---prot}

    El primer paso es generar un archivo tabulado con el siguiente formato,
conteniendo la informacion de la especie, la secuencia de la proteina,
del rna y las secuencias de los hits prot y rna.

    Estos datos son procesados por el siguiente script de python script de
python, que genera las tablas necesarias para ser luego graficadas por
ggplot en R.

    \begin{Shaded}
\begin{Highlighting}[]

\ImportTok{import}\NormalTok{ pandas }\ImportTok{as}\NormalTok{ pd}
\ImportTok{import}\NormalTok{ numpy }\ImportTok{as}\NormalTok{ np}


\CommentTok{#carga como df del archivo de entrada, el cual debe tener el formato:}
\NormalTok{entrada_df}\OperatorTok{=}\NormalTok{ pd.read_table(archivo_entrada)}

\NormalTok{entrada_df.iloc[:,[}\DecValTok{1}\NormalTok{,}\DecValTok{3}\NormalTok{]].}\BuiltInTok{apply}\NormalTok{(f, axis}\OperatorTok{=}\DecValTok{1}\NormalTok{)}

\CommentTok{#agrego cuatro columnas con las posiciones iniciales y finales para los hits de proteina y rna.}
\CommentTok{#primero busca con un rfind, y luego a ese numero le suma el largo del hit.}
\NormalTok{entrada_df[}\StringTok{"prot_hit_s"}\NormalTok{] }\OperatorTok{=}\NormalTok{ entrada_df[[}\StringTok{"prot_seq"}\NormalTok{, }\StringTok{"prot_hit"}\NormalTok{]].}\BuiltInTok{apply}\NormalTok{(}\KeywordTok{lambda}\NormalTok{ x: x[}\DecValTok{0}\NormalTok{].find(x[}\DecValTok{1}\NormalTok{]), axis}\OperatorTok{=}\DecValTok{1}\NormalTok{)}
\NormalTok{entrada_df[}\StringTok{"prot_hit_e"}\NormalTok{] }\OperatorTok{=}\NormalTok{ entrada_df[[}\StringTok{"prot_hit"}\NormalTok{, }\StringTok{"prot_hit_s"}\NormalTok{]].}\BuiltInTok{apply}\NormalTok{(}\KeywordTok{lambda}\NormalTok{ x: }\BuiltInTok{len}\NormalTok{(x[}\DecValTok{0}\NormalTok{]) }\OperatorTok{+}\NormalTok{ x[}\DecValTok{1}\NormalTok{], axis}\OperatorTok{=}\DecValTok{1}\NormalTok{)}
\NormalTok{entrada_df[}\StringTok{"rna_hit_s"}\NormalTok{] }\OperatorTok{=}\NormalTok{ entrada_df[[}\StringTok{"rna_seq"}\NormalTok{, }\StringTok{"rna_hit"}\NormalTok{]].}\BuiltInTok{apply}\NormalTok{(}\KeywordTok{lambda}\NormalTok{ x: x[}\DecValTok{0}\NormalTok{].find(x[}\DecValTok{1}\NormalTok{]), axis}\OperatorTok{=}\DecValTok{1}\NormalTok{)}
\NormalTok{entrada_df[}\StringTok{"rna_hit_e"}\NormalTok{] }\OperatorTok{=}\NormalTok{ entrada_df[[}\StringTok{"rna_hit"}\NormalTok{, }\StringTok{"rna_hit_s"}\NormalTok{]].}\BuiltInTok{apply}\NormalTok{(}\KeywordTok{lambda}\NormalTok{ x: }\BuiltInTok{len}\NormalTok{(x[}\DecValTok{0}\NormalTok{]) }\OperatorTok{+}\NormalTok{ x[}\DecValTok{1}\NormalTok{], axis}\OperatorTok{=}\DecValTok{1}\NormalTok{)}


\CommentTok{#subset y separacion de tabla inicial en proteina y rna}
\NormalTok{prot_df }\OperatorTok{=}\NormalTok{ entrada_df[[}\StringTok{"sp"}\NormalTok{, }\StringTok{"prot_seq"}\NormalTok{, }\StringTok{"prot_hit_s"}\NormalTok{, }\StringTok{"prot_hit_e"}\NormalTok{]]}
\NormalTok{rna_df }\OperatorTok{=}\NormalTok{ entrada_df[[}\StringTok{"sp"}\NormalTok{, }\StringTok{"rna_seq"}\NormalTok{, }\StringTok{"rna_hit_s"}\NormalTok{, }\StringTok{"rna_hit_e"}\NormalTok{]]}



\CommentTok{# ----------------------- #}
\CommentTok{# reformateo los data frame para obtener un df (para cada uno) con el formato necesario para ggplot,}
\CommentTok{# esto es una residuo o nucleotido por fila, y una columna color donde 1 es que esa posicion es un hit.}


\CommentTok{# Proteina}
\NormalTok{prot_df_salida }\OperatorTok{=}\NormalTok{ pd.DataFrame()}
\ControlFlowTok{for}\NormalTok{ i }\KeywordTok{in} \BuiltInTok{range}\NormalTok{(}\BuiltInTok{len}\NormalTok{(prot_df)):}
\NormalTok{    virus }\OperatorTok{=}\NormalTok{ prot_df.iloc[i, }\DecValTok{0}\NormalTok{]}
\NormalTok{    n_virus }\OperatorTok{=}\NormalTok{ i }\OperatorTok{+} \DecValTok{1}
\NormalTok{    seq_list }\OperatorTok{=} \BuiltInTok{list}\NormalTok{(prot_df.iloc[i, }\DecValTok{1}\NormalTok{])}
\NormalTok{    l }\OperatorTok{=} \BuiltInTok{len}\NormalTok{(prot_df.iloc[i, }\DecValTok{1}\NormalTok{])}
\NormalTok{    hit_s }\OperatorTok{=}\NormalTok{ prot_df.iloc[i, }\DecValTok{2}\NormalTok{]}
\NormalTok{    hit_e }\OperatorTok{=}\NormalTok{ prot_df.iloc[i, }\DecValTok{3}\NormalTok{]}
\NormalTok{    x }\OperatorTok{=} \BuiltInTok{range}\NormalTok{(}\DecValTok{1}\NormalTok{,l }\OperatorTok{+} \DecValTok{1}\NormalTok{)}
\NormalTok{    tmp }\OperatorTok{=}\NormalTok{ pd.DataFrame(\{}\StringTok{"x"}\NormalTok{: x, }\StringTok{"sp_id"}\NormalTok{: [n_virus] }\OperatorTok{*}\NormalTok{ l, }\StringTok{"sp"}\NormalTok{: [virus] }\OperatorTok{*}\NormalTok{ l, }\StringTok{"seq"}\NormalTok{: seq_list, }\StringTok{"color"}\NormalTok{: [}\DecValTok{0}\NormalTok{] }\OperatorTok{*}\NormalTok{ l\})}
\NormalTok{    tmp[}\StringTok{"color"}\NormalTok{][hit_s:hit_e] }\OperatorTok{=}\NormalTok{ [}\DecValTok{1}\NormalTok{] }\OperatorTok{*}\NormalTok{ (hit_e }\OperatorTok{-}\NormalTok{ hit_s)}
\NormalTok{    prot_df_salida }\OperatorTok{=}\NormalTok{ pd.concat([prot_df_salida, tmp])}

\CommentTok{# RNA}
\NormalTok{rna_df_salida }\OperatorTok{=}\NormalTok{ pd.DataFrame()}
\ControlFlowTok{for}\NormalTok{ i }\KeywordTok{in} \BuiltInTok{range}\NormalTok{(}\BuiltInTok{len}\NormalTok{(rna_df)):}
\NormalTok{    virus }\OperatorTok{=}\NormalTok{ rna_df.iloc[i, }\DecValTok{0}\NormalTok{]}
\NormalTok{    n_virus }\OperatorTok{=}\NormalTok{ i }\OperatorTok{+} \DecValTok{1}
\NormalTok{    seq_list }\OperatorTok{=} \BuiltInTok{list}\NormalTok{(rna_df.iloc[i, }\DecValTok{1}\NormalTok{])}
\NormalTok{    l }\OperatorTok{=} \BuiltInTok{len}\NormalTok{(rna_df.iloc[i, }\DecValTok{1}\NormalTok{])}
\NormalTok{    l_aa }\OperatorTok{=}\NormalTok{ l }\OperatorTok{/} \DecValTok{3}
\NormalTok{    x_rna }\OperatorTok{=}\NormalTok{ np.array([])}
\NormalTok{    hit_s }\OperatorTok{=}\NormalTok{ rna_df.iloc[i, }\DecValTok{2}\NormalTok{]}
\NormalTok{    hit_e }\OperatorTok{=}\NormalTok{ rna_df.iloc[i, }\DecValTok{3}\NormalTok{]}
    \ControlFlowTok{for}\NormalTok{ e }\KeywordTok{in} \BuiltInTok{range}\NormalTok{(l_aa }\OperatorTok{+} \DecValTok{1}\NormalTok{):}
\NormalTok{        x_rna }\OperatorTok{=}\NormalTok{ np.append(x_rna, np.array([}\FloatTok{1.25}\NormalTok{, }\FloatTok{1.5}\NormalTok{, }\FloatTok{1.75}\NormalTok{]) }\OperatorTok{+}\NormalTok{ e)}
\NormalTok{    x_rna }\OperatorTok{=}\NormalTok{ x_rna[:l]}
\NormalTok{    tmp }\OperatorTok{=}\NormalTok{ pd.DataFrame(\{}\StringTok{"x"}\NormalTok{: x_rna, }\StringTok{"sp_id"}\NormalTok{: [n_virus] }\OperatorTok{*}\NormalTok{ l, }\StringTok{"sp"}\NormalTok{: [virus] }\OperatorTok{*}\NormalTok{ l, }\StringTok{"seq"}\NormalTok{: seq_list, }\StringTok{"color"}\NormalTok{: [}\DecValTok{0}\NormalTok{] }\OperatorTok{*}\NormalTok{ l\})}
\NormalTok{    tmp[}\StringTok{"color"}\NormalTok{][hit_s:hit_e] }\OperatorTok{=}\NormalTok{ [}\DecValTok{2}\NormalTok{] }\OperatorTok{*}\NormalTok{ (hit_e }\OperatorTok{-}\NormalTok{ hit_s)}
\NormalTok{    rna_df_salida }\OperatorTok{=}\NormalTok{ pd.concat([rna_df_salida, tmp])}

\CommentTok{#reordenar las columnas}
\NormalTok{prot_df_salida }\OperatorTok{=}\NormalTok{ prot_df_salida[[}\StringTok{"sp"}\NormalTok{, }\StringTok{"sp_id"}\NormalTok{, }\StringTok{"x"}\NormalTok{, }\StringTok{"seq"}\NormalTok{, }\StringTok{"color"}\NormalTok{]]}
\NormalTok{rna_df_salida }\OperatorTok{=}\NormalTok{ rna_df_salida[[}\StringTok{"sp"}\NormalTok{, }\StringTok{"sp_id"}\NormalTok{, }\StringTok{"x"}\NormalTok{, }\StringTok{"seq"}\NormalTok{, }\StringTok{"color"}\NormalTok{]]}

\NormalTok{prot_df_salida.to_csv(}\StringTok{"/home/fernando/git/flavivirus-disorder/2019/flavivirus/no_X/hit08/visualizador_ali/aa.csv"}\NormalTok{, index}\OperatorTok{=}\VariableTok{False}\NormalTok{)}
\NormalTok{rna_df_salida.to_csv(}\StringTok{"/home/fernando/git/flavivirus-disorder/2019/flavivirus/no_X/hit08/visualizador_ali/rna.csv"}\NormalTok{, index}\OperatorTok{=}\VariableTok{False}\NormalTok{)}
\end{Highlighting}
\end{Shaded}

    carga de los datos en R y grafico:

    \begin{Verbatim}[commandchars=\\\{\}]
{\color{incolor}In [{\color{incolor}9}]:} \PY{k+kn}{library}\PY{p}{(}\PY{l+s}{\PYZdq{}}\PY{l+s}{ggplot2\PYZdq{}}\PY{p}{)}
        \PY{k+kn}{library}\PY{p}{(}\PY{l+s}{\PYZdq{}}\PY{l+s}{repr\PYZdq{}}\PY{p}{)}
        
        \PY{k+kp}{options}\PY{p}{(}repr.plot.width\PY{o}{=}\PY{l+m}{10}\PY{p}{,} repr.plot.height\PY{o}{=}\PY{l+m}{6}\PY{p}{)} \PY{c+c1}{\PYZsh{}tamano grafico salida}
        
        
        df\PYZus{}aa \PY{o}{=} read.csv\PY{p}{(} \PY{l+s}{\PYZdq{}}\PY{l+s}{/home/fernando/git/ag/visualizador\PYZus{}ali/aa.csv\PYZdq{}}\PY{p}{,} stringsAsFactors \PY{o}{=} \PY{n+nb+bp}{F}\PY{p}{)}
        df\PYZus{}rna \PY{o}{=} read.csv\PY{p}{(} \PY{l+s}{\PYZdq{}}\PY{l+s}{/home/fernando/git/ag/visualizador\PYZus{}ali/rna.csv\PYZdq{}}\PY{p}{,} stringsAsFactors \PY{o}{=} \PY{n+nb+bp}{F}\PY{p}{)}
        
        df\PYZus{}aa\PY{o}{\PYZdl{}}color \PY{o}{=} \PY{k+kp}{as.character}\PY{p}{(}df\PYZus{}aa\PY{o}{\PYZdl{}}color\PY{p}{)}
        df\PYZus{}rna\PY{o}{\PYZdl{}}color \PY{o}{=} \PY{k+kp}{as.character}\PY{p}{(}df\PYZus{}rna\PY{o}{\PYZdl{}}color\PY{p}{)}
        etiquetas \PY{o}{\PYZlt{}\PYZhy{}} \PY{k+kp}{unique}\PY{p}{(}df\PYZus{}aa\PY{o}{\PYZdl{}}species\PY{p}{)}
        n \PY{o}{\PYZlt{}\PYZhy{}} \PY{k+kp}{max}\PY{p}{(}df\PYZus{}aa\PY{o}{\PYZdl{}}sp\PYZus{}id\PY{p}{)}
        
        ggplot\PY{p}{(}\PY{p}{)} \PY{o}{+}
          geom\PYZus{}text\PY{p}{(}data\PY{o}{=}df\PYZus{}aa\PY{p}{,} aes\PY{p}{(}x\PY{o}{=}x \PY{o}{+} \PY{l+m}{0.5}\PY{p}{,} y\PY{o}{=} sp\PYZus{}id\PY{p}{,} label\PY{o}{=}\PY{k+kp}{seq}\PY{p}{,} color\PY{o}{=} color\PY{p}{)}\PY{p}{,} size\PY{o}{=}rel\PY{p}{(}\PY{l+m}{11}\PY{p}{)}\PY{p}{,} family\PY{o}{=}\PY{l+s}{\PYZdq{}}\PY{l+s}{mono\PYZdq{}}\PY{p}{)} \PY{o}{+} \PY{c+c1}{\PYZsh{} grafica secuencia proteina}
          geom\PYZus{}text\PY{p}{(}data\PY{o}{=}df\PYZus{}rna\PY{p}{,} aes\PY{p}{(}x\PY{o}{=}x\PY{p}{,} y\PY{o}{=}sp\PYZus{}id \PY{o}{\PYZhy{}}  \PY{l+m}{0.4} \PY{p}{,} label\PY{o}{=}\PY{k+kp}{seq}\PY{p}{,} color\PY{o}{=} color\PY{p}{)}\PY{p}{,} size\PY{o}{=}rel\PY{p}{(}\PY{l+m}{4}\PY{p}{)}\PY{p}{,} family\PY{o}{=}\PY{l+s}{\PYZdq{}}\PY{l+s}{mono\PYZdq{}}\PY{p}{)} \PY{o}{+} \PY{c+c1}{\PYZsh{} grafica secuencia rna}
          scale\PYZus{}color\PYZus{}manual\PY{p}{(} values\PY{o}{=}\PY{k+kt}{c}\PY{p}{(}\PY{l+s}{\PYZdq{}}\PY{l+s}{gray51\PYZdq{}}\PY{p}{,} \PY{l+s}{\PYZdq{}}\PY{l+s}{blue\PYZdq{}}\PY{p}{,} \PY{l+s}{\PYZdq{}}\PY{l+s}{red\PYZdq{}}\PY{p}{)}\PY{p}{)} \PY{o}{+} \PY{c+c1}{\PYZsh{} gis para las secuencias que no covarian, azul para cov en prot y rojo en rna}
          theme\PY{p}{(}panel.background\PY{o}{=}element\PYZus{}rect\PY{p}{(}fill\PY{o}{=}\PY{l+s}{\PYZdq{}}\PY{l+s}{white\PYZdq{}}\PY{p}{,} colour\PY{o}{=}\PY{l+s}{\PYZdq{}}\PY{l+s}{white\PYZdq{}}\PY{p}{)}\PY{p}{,}
                axis.title \PY{o}{=} element\PYZus{}blank\PY{p}{(}\PY{p}{)}\PY{p}{,}
                axis.ticks.y \PY{o}{=} element\PYZus{}blank\PY{p}{(}\PY{p}{)}\PY{p}{,}
                axis.text.y \PY{o}{=} element\PYZus{}text\PY{p}{(}family\PY{o}{=}\PY{l+s}{\PYZdq{}}\PY{l+s}{mono\PYZdq{}}\PY{p}{,} size\PY{o}{=}rel\PY{p}{(}\PY{l+m}{2}\PY{p}{)}\PY{p}{)}\PY{p}{,}
                axis.text.x \PY{o}{=} element\PYZus{}text\PY{p}{(}size\PY{o}{=}rel\PY{p}{(}\PY{l+m}{0.7}\PY{p}{)}\PY{p}{)}\PY{p}{,}
                legend.text \PY{o}{=} element\PYZus{}text\PY{p}{(}size\PY{o}{=}rel\PY{p}{(}\PY{l+m}{0.7}\PY{p}{)}\PY{p}{)}\PY{p}{,}
                legend.key.size \PY{o}{=} unit\PY{p}{(}\PY{l+m}{0.7}\PY{p}{,} \PY{l+s}{\PYZdq{}}\PY{l+s}{lines\PYZdq{}}\PY{p}{)}\PY{p}{,}
                legend.position \PY{o}{=} \PY{l+s}{\PYZdq{}}\PY{l+s}{none\PYZdq{}}\PY{p}{)} \PY{o}{+}
          scale\PYZus{}y\PYZus{}continuous\PY{p}{(}breaks\PY{o}{=} \PY{l+m}{1}\PY{o}{:}n\PY{p}{,} labels \PY{o}{=} etiquetas\PY{p}{)} \PY{o}{+}
          geom\PYZus{}vline\PY{p}{(}data\PY{o}{=}df\PYZus{}aa\PY{p}{,}aes\PY{p}{(}xintercept \PY{o}{=}x\PY{p}{)}\PY{p}{,}linetype\PY{o}{=}\PY{l+s}{\PYZdq{}}\PY{l+s}{dotted\PYZdq{}}\PY{p}{,}   color \PY{o}{=} \PY{l+s}{\PYZdq{}}\PY{l+s}{gray70\PYZdq{}}\PY{p}{,} size\PY{o}{=}\PY{l+m}{0.5}\PY{p}{)} \PY{o}{+} \PY{c+c1}{\PYZsh{} linea para visualizar mejor las columnas}
          ggtitle\PY{p}{(}\PY{l+s}{\PYZdq{}}\PY{l+s}{MBF\PYZdq{}}\PY{p}{)}
\end{Verbatim}


    \begin{center}
    \adjustimage{max size={0.9\linewidth}{0.9\paperheight}}{output_5_0.png}
    \end{center}
    { \hspace*{\fill} \\}
    

    % Add a bibliography block to the postdoc
    
    
    
    \end{document}
